\documentclass[paper=a4,10pt]{scrartcl}

\usepackage[utf8x]{inputenc}
\usepackage[ngerman]{babel}
\usepackage[T1]{fontenc}

\usepackage{graphicx}
\usepackage{float}
\usepackage{subcaption}

\usepackage{fancyref}

\usepackage[numbers,square,sort]{natbib} %praktikumsquellenvorgabe
\usepackage{amsmath}
\usepackage{amssymb}

\usepackage{bm}

\usepackage{url}
\usepackage{hyperref}

\usepackage[a4paper, includehead, includefoot]{geometry}
\geometry{left=2cm, right=2cm, top=2cm, bottom=2cm}

\DeclareMathOperator*{\argmax}{argmax}
\DeclareMathOperator*{\argmin}{argmin}

\begin{document}

\title{Altarelli Zsmfassung}
%\author{Author's Name}

\section{grober Ablauf}
Es gilt in einem bipartiten Graphen $(L,R,E)$ ein maximum Matching zu finden. $L$ ist nochmal aufgeteilt in $L_1$ und $L_2$. Ein Knoten in $L_2$ will be available for matching with given prob $p_{l_2}$. 

In der ersten Stage werden die Knoten in $L_1$ gematcht nur mit den Informationen, die durch die Wahrscheinlichkeiten gegeben sind. In der zweiten Stage werden entsprechend $p$ die verfügbaren Knoten aus $L_2$ gewonnen und gematcht.

Und das Ziel ist es, die Größe des finalen Matchings zu maximieren.

\section{genauer}
Es werden zwei Mengen eingeführt: 
\begin{align}
\bm x_1 &= \{ x_{{l_1}r} \in \{0,1\} | \{ l_1,r\} \in E, l_1\in L_1 \} \\
\bm x_2 &= \{ x_{{l_2}r} \in \{0,1\} | \{ l_2,r\} \in E, l_2\in L_2 \}
\end{align}
mit $x_{lr}=1$ iff (kp wieso nicht einfach nur if) $\{ l,r\} \in M \subset E$. Damit wird also angezeigt, dass sich eine Kante im Matching befindet oder nicht. Die $x_{lr}$ sind nur dann 1, wenn diese Kante im Matching enthalten ist.
Zusätzlich gibt es jetzt noch:
\begin{align}
t=\{t_{l_2} \in \{0,1\}| l_2\in L_2\}
\end{align}
mit $t_{l_2}=1$ iff $l_2$ is available for matching in the second stage. \textit{Also nachdem entsprechend der Wahrscheinlichkeiten ausgewählt wurde?}

Jetzt mit der cost function $\mathcal{E} (x_1,x_2,t)$, die von den noch verfügbaren Knoten diejenigen zählt, die noch nicht gematcht sind, finde:
\begin{align}
x_1^* = \argmin_{x_1} \mathbb{E}_t \min_{x_2} \mathcal{E}(x_1,x_2,t).
\end{align}
Ich möchte also das $x_1$ wählen, welches den Teil $\mathbb{E}_t \min_{x_2} \mathcal{E}(x_1,x_2,t)$ minimiert. Dieser Teil ist aber nichttrivial stark von $x_1$ abhängig und das ist das Hauptproblem.

\textit{Wenn ich die Größe des finalen Matchings maximieren möchte muss ich dafür sorgen, dass möglichst wenig Knoten übrig bleiben die nicht mehr dem Matching hinzugefügt werden können. Die cost function verstehe ich so, dass sie am Ende zählt, wie viele Knoten nicht gematcht sind. Wenn $x_1$ feststeht und die $t_{l_2}$ bestimmt sind, ist es straighforward to find the optimal $x_2$ (weil das dann einfach nur ein normales Matchingproblem ist?).}

Dazu gibt es noch Constraints. Es wird die Hilfsmenge $\partial r = \{l\in L | \{l,r\} \in E \}$, welche alle Knoten in $L$ enthält, die mit dem Knoten $r\in R$ durch eine Kante in $E$ verbunden sind.:
\begin{itemize}
\item constraint 1:
 \begin{align}
\sum_{l\in \partial r} x_{lr} \le 1 \quad \forall r\in R
\end{align} 
Für jeden Knoten in $R$ darf es im Matching also nur maximal einen Knoten in $L$ geben, der mit $r$ gematcht ist. Diese Beschränkung setzt also die Definition des Matchings um. 
\item constraint 2:
 \begin{align}
\sum_{r\in \partial l_1} x_{l_1r} \le 1 \quad \forall l_1\in L_1
\end{align} 
Das selbe wie eben soll jetzt auch für alle Knoten aus $L_1$ gelten: Es darf maximal eine Kante im Matching von den Knoten $l_1$ abgehen.
\item constraint 3:
\begin{align}
\sum_{r\in \partial l_2} x_{l_2r} \le t_{l_2} \quad \forall l_2\in L_2
\end{align}
Das sagt aus, dass falls ein Knoten aus $L_2$ am Ende nicht durch die Wahrscheinlichkeit realisiert wird, dann darf er auch in keiner Kante im Matching auftauchen. Falls er realisiert wird dann gilt das selbe wie für die anderen Knoten und er darf nur in einer Kante vorkommen. 
\end{itemize}

\section{Fragen}
\subsection{Offen: Werden aus $L_1$ alle Knoten in der ersten stage gematcht?}
Kann es sein, dass die Wahrscheinlichkeiten nahelegen, dass bestimmte Knoten lieber nicht gematcht werden?

Das zweite constraint schreibt auf jeden Fall nicht vor, dass alle Knoten aus $L_1$ gematcht werden müssen, es kann auch Knoten geben von denen keine Kante im Matching berücksichtigt wird.

Also würde ich eher nein auf die Frage antworten. 

\subsection{Offen: zur cost function}
Gehört mit der ersten Frage zusammen.\\
Die cost function verstehe ich so, dass sie am Ende zählt, wie viele Knoten nicht gematcht sind. Im Text steht aber: ``counting the number of umatched vertice samong the available ones''. Zählt die also nur die Knoten für die $t_{l_2}=1$ war und die am Ende übrig bleiben? Dann würden ja Knoten, die aus $L_1$ übrig geblieben sind nicht mitgezählt. 

\subsection{Offen: Ist das jetzt ein lineares Programm?}
Und wenn, wieso gibt es das 3. constraint, das sich auf die $x_2$ bezieht, obwohl wir nach einem $x_1$ suchen. 



\end{document}