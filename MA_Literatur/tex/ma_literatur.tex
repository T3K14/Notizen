\documentclass[paper=a4,10pt]{scrartcl}

\usepackage[utf8x]{inputenc}
\usepackage[ngerman]{babel}
\usepackage[T1]{fontenc}

\usepackage{graphicx}
\usepackage{float}
\usepackage{subcaption}

\usepackage{fancyref}

\usepackage{amsmath}
\usepackage{amssymb}

\usepackage{hyperref}

\usepackage[a4paper, includehead, includefoot]{geometry}
\geometry{left=2cm, right=2cm, top=2cm, bottom=2cm}

\begin{document}

\title{MA-Logbuch}
%\author{Author's Name}

\section{Paper}
\subsection{STOCHASTIC SPANNING TREE PROBLEM}
\paragraph{Status} Noch nicht angeschaut
\paragraph{Verzeichnis} paper/von mir rausgesucht/ altes stochastic spanning tree

\subsection{Cavity Method:  Message Passing from a Physics Perspective}

Enthält Notizen einer Vorlesung von Marc Mezard und ist wohl physikalischer motiviert.

\paragraph{Status} Noch nicht angeschaut
\paragraph{Verzeichnis} paper/von mir rausgesucht/cavity method

\subsection{The physical Meaning of Replica Symmetry Breaking}

physikalischer Ansatz zu Replica Symmetry breaking

\paragraph{Status} Noch nicht angeschaut
\paragraph{Verzeichnis} paper/von mir rausgesucht/replica sym breaking

\subsection{The number of matchings in randomgraphs}

ist auch von Mezard

\paragraph{Status} Noch nicht angeschaut
\paragraph{Verzeichnis} paper/von mir rausgesucht/Zdeborova\_2006\_J.\_Stat.\_Mech.\_2006\_P05003

\subsection{Analytic and Algorithmic Solution of Random Satisfiability Problems}

ist auch von Mezard und anscheinend das Paper wo SP introduced wurde? und damit ``revolutionaer''

\paragraph{Status} Noch nicht angeschaut
\paragraph{Verzeichnis} paper/von mir rausgesucht/Analytic and AlgorithmicSolution of RandomSatisfiability Problems

\section{Bücher}
\subsection{Integer Programmingand CombinatorialOptimization}

Enthält zu dem Vortrag zu MST das Grundlagenpaper ``On Two-Stage Stochastic Minimum Spanning Trees'' ab Seite 330

\paragraph{Status} Noch nicht angeschaut
\paragraph{Verzeichnis} paper/von mir rausgesucht/2005\_Book\_IntegerProgrammingAndCombinato

\subsection{towards new, statistical-mechanics motivated algorithms}
Kapitel aus Hartmanns Buch über BP und SP

\paragraph{Status} Noch nicht angeschaut
\paragraph{Verzeichnis} paper/chapter\_message\_passing

\section{Vorträge}



\end{document}