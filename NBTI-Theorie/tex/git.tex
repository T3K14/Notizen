\documentclass[paper=a4,10pt]{scrartcl}

\usepackage[utf8x]{inputenc}
\usepackage[ngerman]{babel}
\usepackage[T1]{fontenc}

\usepackage{graphicx}
\usepackage{float}
\usepackage{subcaption}

\usepackage{fancyref}

\usepackage[numbers,square,sort]{natbib} %praktikumsquellenvorgabe
\usepackage{amsmath}
\usepackage{amssymb}

\usepackage{url}
\usepackage{hyperref}

\usepackage{bm}

\usepackage[a4paper, includehead, includefoot]{geometry}
\geometry{left=2cm, right=2cm, top=2cm, bottom=2cm}

\begin{document}

\title{NBTI-Theorie}
%\author{Author's Name}

\section{Grundlagen}
\subsection{Basic Band Theory}
Fangen an mit freiem Elektronengasmodel mit nur einem Elektron in einem Potentialtopf der breite $L$, welcher den Kristall darstellen soll. In diesem Fall kann anscheinend die Energie des Elektrons nur vollständig in kinetischer Form vorliegen. Jetzt wird die zeitunabhängige Schrödingergleichung  mit Potential gleich 0 und periodischen Randbedingungen gelöst zu:
\begin{align}
\psi = \left( \frac{1}{L}	\right)^{3/2} \cdot e^{i  \bm k \bm r}
\end{align}
mit $k_x = \pm n_x \cdot 2\pi / L$ mit der Quantenzahl $n_x=0,1,2,\dots$ und genauso für die anderen beiden Komponenten von $\bm k$.\\
Für die Geschwindigkeit des Elektrons erhält man 
\begin{align}
\bm v = \frac{\bm p}{m_e} = \frac{\hbar \cdot \bm k}{m_e}.
\end{align}
Da die total Energy hier identisch mit $E_{\text{kin}}$ ist, gilt
\begin{align}
E = \frac{m_e v^2}{2} = \frac{\hbar^2 \bm k^2}{2m_e} = \frac{\hbar^2}{2m_e} \left( \frac{2\pi}{L} \right)^2 (n_x^2 +n_y^2 + n_z^2)
\end{align}


\section{Erklärungen}
\subsection{Bias}
\textit{Bisher habe ich das so verstanden, dass das einfach die Spannung ist, die zwischen Source und Gate angelegt ist.} \\

\noindent
In einem Paper über NBTI (Statistical Model for MOSFET Bias TemperatureInstability Component Due to Charge Trapping von G. Wirth) steht, dass bei konst. Bias nur states (traps) nahe des Fermilevels eine starke Aktivität aufweisen, also ihren Zustand zwischen besetzt und leer wechseln. Das kann man als steady state condition interpretieren -> occupation probability of a trap is time independent -> Anzahl an Traps die als besetzt erwartet werden ist konstant. Falls der bias point sich abrupt ändert, ändert sich auch die Besetzungswahrscheinlichkeit abrupt.
\textit{Intuitiv kann ich das nicht einordnen.}

\subsection{threshold voltage}
Das ist die minimale Gate-Source Spannung, die angelegt sein muss, damit eine leitende Brücke zwischen source und drain entsteht.


\end{document}