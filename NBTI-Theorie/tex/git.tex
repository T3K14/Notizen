\documentclass[paper=a4,10pt]{scrartcl}

\usepackage[utf8x]{inputenc}
\usepackage[ngerman]{babel}
\usepackage[T1]{fontenc}

\usepackage{graphicx}
\usepackage{float}
\usepackage{subcaption}

\usepackage{fancyref}

\usepackage[numbers,square,sort]{natbib} %praktikumsquellenvorgabe
\usepackage{amsmath}
\usepackage{amssymb}

\usepackage{url}
\usepackage{hyperref}

\usepackage{bm}

\usepackage[a4paper, includehead, includefoot]{geometry}
\geometry{left=2cm, right=2cm, top=2cm, bottom=2cm}

\begin{document}

\title{NBTI-Theorie}
%\author{Author's Name}

\section{Grundlagen}
\subsection{Basic Band Theory}
Fangen an mit freiem Elektronengasmodel mit nur einem Elektron in einem Potentialtopf der breite $L$, welcher den Kristall darstellen soll. In diesem Fall kann anscheinend die Energie des Elektrons nur vollständig in kinetischer Form vorliegen. Jetzt wird die zeitunabhängige Schrödingergleichung  mit Potential gleich 0 und periodischen Randbedingungen gelöst zu:
\begin{align}
\psi = \left( \frac{1}{L}	\right)^{3/2} \cdot e^{i  \bm k \bm r}
\end{align}
mit $k_x = \pm n_x \cdot 2\pi / L$ mit der Quantenzahl $n_x=0,1,2,\dots$ und genauso für die anderen beiden Komponenten von $\bm k$.\\
Für die Geschwindigkeit des Elektrons erhält man 
\begin{align}
\bm v = \frac{\bm p}{m_e} = \frac{\hbar \cdot \bm k}{m_e}.
\end{align}
Da die total Energy hier identisch mit $E_{\text{kin}}$ ist, gilt
\begin{align}
E = \frac{m_e v^2}{2} = \frac{\hbar^2 \bm k^2}{2m_e} = \frac{\hbar^2}{2m_e} \left( \frac{2\pi}{L} \right)^2 (n_x^2 +n_y^2 + n_z^2)
\end{align}
Das ist eine Dispersionsrelation, weil $E$ durch $\bm k$ ausgedrückt wird.
Diese ganze Herleitung ist dazu da gewesen, um zu zeigen, dass die Energiewerte nur diskret sein können. Dieses Ergebnis ist auch dann noch gültig, wenn man mit den korrekten Potentialen rechnet und mehrere Elektronen betrachtet. Der Zusammenhang zwischen $E$ und $\bm k$ wird dann jedoch deutlich komplexer.\\
Jetzt wo die Energielevel bekannt sind, kann man zählen, wie viele Energielevel es in einem Energieintervall $\Delta E$ bei der Energie $E$ gibt. Das macht man im $k$-Raum, im phase space.

Im phase space ist eine Fläche konstanter Ebene eine Sphäre, da alle Punkte, deren $\bm k$ die selbe Länge haben auch den selben Energiewert besitzen.
Jeder Zustand (jede Lsg. der Schrödingergleichung) mit einem bestimmten $\bm k$ nimmt ein Volumen ein (kleine Würfel).
Die Anzahl of cubes die in die Sphäre zur Energie $E$ passen ist die Zahl aller Energiezustände bis zu $E$.

Jetzt wollen wir die density of states $D(E)$ ermitteln.

Ein einzelner Zustand nimmt im $\bm k$-Raum ein Volumen von $(2\pi)^3/V$ mit $V=L_xL_yL_z$ und mit dem Faktor 2 für die Spinentartung hat man einen Zustand pro diesem Volumen -> $Z(\bm k) = 2 \cdot V/(2\pi)^3$. 

Die Zustandsdichte im Energieraum erhält man mit der Dispersionsrelation von oben und der Beziehung
\begin{align}
Z(\bm k) d^3k = D(E)dE.
\end{align} 
\textit{Wo kommt das her?}

Die Herleitung im Marx ist für mich immer durch irgendwelche Approximationen nicht konkret genug. Jedoch kommt dabei das überall stehende Ergebnis mit $D(E) \sim \sqrt{E}$ raus.


\section{Erklärungen}
\subsection{Bias}
\textit{Bisher habe ich das so verstanden, dass das einfach die Spannung ist, die zwischen Source und Gate angelegt ist.} \\

\noindent
In einem Paper über NBTI (Statistical Model for MOSFET Bias TemperatureInstability Component Due to Charge Trapping von G. Wirth) steht, dass bei konst. Bias nur states (traps) nahe des Fermilevels eine starke Aktivität aufweisen, also ihren Zustand zwischen besetzt und leer wechseln. Das kann man als steady state condition interpretieren -> occupation probability of a trap is time independent -> Anzahl an Traps die als besetzt erwartet werden ist konstant. Falls der bias point sich abrupt ändert, ändert sich auch die Besetzungswahrscheinlichkeit abrupt.
\textit{Intuitiv kann ich das nicht einordnen.}

\subsection{threshold voltage}
Das ist die minimale Gate-Source Spannung, die angelegt sein muss, damit eine leitende Brücke zwischen source und drain entsteht.


\end{document}