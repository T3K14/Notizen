\documentclass[paper=a4,10pt]{scrartcl}

\usepackage[utf8x]{inputenc}
\usepackage[ngerman]{babel}
\usepackage[T1]{fontenc}

\usepackage{graphicx}
\usepackage{float}
\usepackage{subcaption}

\usepackage{fancyref}

\usepackage[numbers,square,sort]{natbib} %praktikumsquellenvorgabe
\usepackage{amsmath}
\usepackage{amssymb}

\usepackage{url}
\usepackage{hyperref}

\usepackage[a4paper, includehead, includefoot]{geometry}
\geometry{left=2cm, right=2cm, top=2cm, bottom=2cm}

\begin{document}

\title{Kombinatorik}
%\author{Author's Name}

\section{Binomialkoeffizient}
\begin{align}
\binom{n}{k} = \frac{n!}{(n-k)!\cdot k!}
\end{align}
Er sagt aus, auf wie viele verschiedene Weisen man $k$ unterscheidbare Objekte aus $n$ ohne Zurücklegen ziehen kann (Anzahl der Kombinationen ohne Wiederholung).
\subsection*{Herleitung}
Wenn man sich zunächst überlegt, wie viele $k$-Tupel man aus $n$ Objekten ziehen kann, dann hat man für das erste $n$ Möglichkeiten, für das zweite $n-1$ und so weiter, bis $n-k+1$. Das zusammengefasst ist $\frac{n!}{(n-k)!}$. Für jedes dieser $k$-Tupel gibt es jetzt selbst wieder $k!$ unterschiedliche Anordnungsmöglichkeiten, da es hier nur um die Menge und nicht um die Reihenfolge geht. Man hat also k! mal weniger Möglichkeiten und teilt daher noch durch $k!$

\subsection*{Beispiel}
Wie viele Möglichkeiten gibt es, 6 Zahlen aus 49 auszuwählen? Es sind $\binom{49}{6} = 13.983.816$.

\end{document}