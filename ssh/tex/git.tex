\documentclass{article}
\usepackage{graphicx}
\usepackage[utf8]{inputenc}
\begin{document}

\title{ssh}
%\author{Author's Name}

\section*{bisheriges Wissen}
Die Keys befinden sich im users/robert/.ssh Ordner. Ein passphrase ist nötig.

\section*{ssh-agent}
Scheinbar muss dieses Programm aktiviert sein. Ich starte es in der git-bash mit \texttt{eval \$(ssh-agent -s)}, alternativ könnte man auch die bashrc so umschreiben, dass das Programm jedes mal beim launch der shell mitgelauncht wird, habe ich aber jetzt nicht gemacht.

Der ssh-agent läuft beyond the duration of a local login session und kommuniziert mit ssh clients.

\section*{Github}
Auf Github habe ich jetzt den neuen ssh-key angegeben und konnte jetzt das repo Notizen auch erfolgreich clonen.

\end{document}